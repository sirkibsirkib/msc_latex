\chapter{Generating Static Governors}
\label{sec:api}
A protcol's \textit{governor} acts to ensure that all the actions of a component are \textit{adherent} to the protocol with which it interfaces, guaranteeing that its actions will not violate the protocol. In this section, we develop a means of embedding governors into Rust's affine type system. As a result, an application developer may ergonomically opt-into checking protocol adherence of their own compute-code using their local Rust compiler, whereafter successful compilation guarantees adherence to the protocol.

In more precise terms, let protocol $A$ be \textit{protocol adherent} to protocol $B$ if and only if the \textit{synchronous composition} of $A$ and $B$ is language-equivalent to $B$; equivalently, $A$ is adherent to $B$ and $A$ adheres to $B$. This can be understood as $A$ contributing no constraints to the composed system that $B$ did not have already.

\section{The Problem: Unintended Constraints}
\label{sec:unintended_constraints}
A central tenet of Reo's design is the \textit{separation of concerns}, part of which is the desire to minimize the knowledge a compute component must have of its protocol. In this view, coordinating the movements of data is not a concern relevant to the task of computation. A desirable balance is possible with the observation that protocol objects are able to partially impose protocol adherence on their neighbors; External ports may instigate a \code{put} or \code{get} at any moment, and the \textit{coordinator} will complete them as soon as the protocol definition allows it. In this way, coordinators possess a crucial subset of the features of \textit{governers}: aligning the \textit{timing} of two actions that compose an interaction. Unfortunately, in the properties of the realm of sequential, action-centric programming itself \textit{implicitly} imposes constraints on the behavior of the system: \code{put} or \code{get} block until their interaction is completed, and no subsequent code (potentially, other port operations) will occur until they do. This is beyond the capabilities of the coordinator to influence.

In the context of application development, this has an interesting consequence; the behavior of the system is influenced by the behavior of (potentially) all of its components. This is sensible in theory, but becomes unwieldy in practice. Even small changes to the behavior of a compute component influences the system's behavior in unexpected ways, as we are not used to thinking about synchronous code as a composable protocol, nor are we able to intuit the \textit{outcome} of the composition. For example, Listing~\ref{listing:transform_not} gives the definition of a compute function which a user may write to interact with a protocol. When \code{p} and \code{g} are connected to a \textit{fifo1} protocol (which forwards \code{p} to \code{g}, buffering it asynchronously in-between), it runs forever and the output will be something like:
\begin{verb}
	I saw true. I saw false. I saw true. I saw false. (...)
\end{verb}.\\However, when connected with the \textit{sync} protocol (which forwards \code{p} to \code{g} synchronously), the system has no behavior. The problem is that even though \textit{fifo1} and \textit{sync} have the same \textit{interface}, \code{transform\_not} is \textit{compatible} (can be made to adhere) with the former and not the latter. By definition, \textit{sync} fires when \textit{both} \code{p} and \code{g} are ready, but \code{transform\_rot} does not \code{put} until the \code{get} is completed. Once the intricacies of these programs grow beyond a programmer's ability to keep track of these relationships, the composed system may have \textit{unintended} behavior. This property may be obvious at the small scale of this example, but it becomes more difficult the larger and more complex the program becomes. In the worst cases, an innocuous change makes an interaction becoming unreachable, manifesting at runtime as \textit{deadlock}. 


\begin{listing}[ht]
	\inputminted[]{rust}{transform_not.rs}
	\caption[Rust example of a compute component.]{A function in Rust which can be used as a compute component in a system, connected to a protocol component.}
	\label{listing:transform_not}
\end{listing}


%That said, the Reo compiler aims to generate code which describes \textit{eac}
%
%In this section
%
%1. intro to the problem at high level.
%2. you are writing a more complicated component. multiple ports yeah!
%3. oh no its not behaving as expected!
%4. section guide

\section{Governors Defined}
\label{sec:governor}
In this work, we accept that it is necessary to write compute code that has blocking behavior. Rather than attempting to empower the coordinator with the ability to further influence its boundary components, we introduce explicit governors into our applications such that from the protocol's perspective, the components appear to manage themselves. A particular compute component requires a particular governor as the behavior permitted to the compute component is a function of its \textit{interface} with the protocol.

Ultimately, all governors have in common that they enforce adherence to a given protocol on the components they govern. However, governors may differ on \textit{when} and \textit{how} this enforcement manifest. For example, a governor may intercept and filter network messages at runtime, while another checks for deviations \textit{statically} and emitting compiler errors.

In this work, we leverage the unique expressiveness of the Rust language by creating a tool which generates protocol-specific governor code. When used by an application developer, these governors assess the protocol adherence of compute functions \textit{statically}, and prevent compilation if deviations are detected. As such, these governor are absent from the compiled binary.

\section{Solution: Static Governance with Types}
\hl{TODO}

\section{Making it Functional}
This section details the workings of the \textbf{Governor generator} tool which generates Rust code given (a) a representation of a protocol's RBA, and (b) the set of ports which comprise the interface of the compute component to be governed.

\subsection{Encoding CA and RBA as Type-State Automata}
% How to do type-state pattern for CA and RBA
\label{sec:type_space_automaton}
The \textit{type state} pattern described in Section~\ref{sec:type_state} provides a means of encoding finite state machines as affine types. Their utility is in guaranteeing that all runtime traces of the resulting program correspond to runs in the automaton. For this class of machines, the encoding is very natural, as there can be a one-to-one correspondence between the states of the abstract automaton, and the types required to represent them. This is also the case for transitions and functions; in the worst case, this mapping is one-to-one also. For an arbitrary transition from states $a$ to $b$ with label $x$, a function can be declared to consume the type for $a$, return the type for $b$, and perform the work associated with $x$ in its body.

The encoding is more complicated for CA, where not only states but data constraints must be encoded into types and must interact with transitions. One approach is to treat \textit{configurations} as \textit{states} were treated before by enumerating them into types. For example, the configuration of state $q_0$ with memory cell $m=0$ is represented by type \code{q\_0\_0}, while state $q_0$ with $m=1$ is represented by \code{q\_0\_1}. On a case-by-case basis, one might be able to represent several configurations using one type in the event these configurations are never \textit{distinguished}. For example, a connector may involve positive integers, but only distinguish their values according to whether they are \textit{odd} or \textit{even} and nothing else; in this case \{\code{q\_0\_0}, \code{q\_0\_1}, \code{q\_0\_2}, ...\} may be collapsed to \{\code{q\_0\_odd}, \code{q\_0\_even}\}. For an arbitrary case unique types are needed for every combination of state with every value of every variable. As RBAs are instances of CA, we are able to represent them using the same procedure. As RBAs are used by both the Reo compiler and Reo-rs, they are the model of choice for governors also. 

\subsection{Rule Consensus}
% how to represent choice
\label{sec:rule_consensus}
The protocol works by firing rules at runtime which correspond to those of the RBA which defines its Reo connector. Section~\ref{sec:type_space_automaton} above explains how various compute-components are able to proceed in lockstep with the protocol's RBA in a type-state automaton of their own. For deterministic RBAs, this is easy enough; everyone can trivially know which action occurs \textit{next}, and they can transition through configuration space independently, safe in the knowledge that their representations of the run will stay aligned. This ignores temporary mis-alignments in \textit{time} for transitions in which the compute component does not communicate with the protocol; for these cases, one may work head, leaving the other in a previous state. However, they will catch up eventually when they both reach a transition that involves them communicating (which is ultimately all that matters). This process becomes more complicated when the protocol can reach configurations with \textit{multiple} choices for the next transitions. Without a priori agreement on how these situations are handled, the choice is defined to be nondeterministic. Clearly, all is well as long as all parties agree on this choice; problems only present themselves when compute components and protocols diagree on what may happen next. If the view of a governor are out of sync with its protocol it is generally unable to guarantee that the actions it permits are adherent, or it may prohibit something it ought not to, resulting in unintended constraints of the intended behavior (in the worst case, deadlock). This is a problem of \textit{consensus}. 

  

%Thus far, we have reasoned about operations on RBAs that preserve their ability to simulate the non-silent port actions of the original. However, at runtime the protocol's state will follow a \textit{particular} path, which may not be the only one possible. When two distinct paths branch out from the current configuration, which one should the governor follow such that it can enforce the correct actions? Consider the protocol \textit{fifo2} once again, and observe that from type-state \code{(E,F)}, two rules may be fired next, one firing $A$ and the other firing $B$. The governor must enforce $A$ if and only if the protocol's state goes down the path for $A$, and likewise for $B$. This is an instance of the \textit{consensus} problem; all RBAs in the system must agree on the path taken such that they can proceed in lockstep.

Many means of creating consensus exist. We are able to enforce a \textit{meta-protocol} a priori between governors and protocol such that consensus emerges at runtime. This can be achieved without any overhead by making the decision based only on information statically available. For example, peers may rely on a shared, total \textit{priority} ordering on rules to remove all nondeterministic choice. Many such meta-protocols are possible, each making assumptions about the desired system behavior.

This work takes the approach of statically `electing' the \textit{coordinator} as the leader in every case, and having all governors follow the lead of its arbitrary choice by `asking' it what to do next \textit{dynamically}. This approach is primarily motivated by its \textit{flexibility}; by supporting an arbitrary choice on the part of the protocol, we make the choice itself an \textit{orthogonal} concern for future work. Electing the coordinator in particular as the leader is also somewhat natural; It is only actor in the system with a complete view of the protocol's state, allowing it to make the choice as a function of the state, ie.\ the protocol is capable of making the best-informed decisions.

In terms of implementation, we make a modification to the encoding of our governor's automaton such that it can represent all choices available from a particular configuration. Before proceeding, the governor `collapses' these options to match the choice of the protocol by communicating with the coordinator. Concretely, the Rust function for a rule no longer returns a \textit{particular} type-state token, but rather a \code{StateSet} which enumerates the options. This object is collapsed as a result of calling \code{determine}. Handling the returned variants branches the governor's control flow in a manner akin to a \code{match} (similar to a \code{switch} statement in other languages), with each arm given a single state token to proceed. Na\"ively, this must be encoded as a distinct \code{enum} type with a variant for every possible outcome. Clearly creating new type definitions for every conceivable combination of branches is prohibitively expensive\footnote{Early versions of our implementation indeed enumerated these types with a relatively effective \textit{powerset} construction. However, it was unable to avoid explosion if there simply were many solutions to be found. The nail in the coffin was the changing from the exponential base from~2 to~3 as a result of the modification explained in Section~\ref{sec:user_defined_simplification}.}. Ideally we wish to be able to create enumeration types on-the-fly with precisely the variants needed on a case-by-case basis. Rust provides \textit{tuples} for this purpose in the case of \code{struct} (product types), it has no parallel for \code{enum} (sum types). This feature has been requested for some time~\cite{anon_sum}. If it is supported one day, this may be ideal fore representing these \code{StateSets} with minimal code generation. 

Until anonymous sum types are supported, our solution to the problem of representing the generic \code{StateSet} type relies on Rust's traits to encode the variants as a \textit{tail-recursive} list of nested tuples. Matching the elements of the lists is achieved by repeatedly attempting to match the \textit{head}. Listing~\ref{listing:state_set} shows one possible representation which uses a final \textit{sentinel} list element to make for an ergonomic definition of the \code{StateSetMatch} trait, which provides distinct head-matching behavior for \textit{singleton} lists from that of larger ones. From the user's perspective, \code{StateSet} objects are opaque, and prevent the automaton from proceeding with transitions until the object is collapsed to some usable \code{State} object.

\begin{listing}[h!]
	\centering
	\inputminted{rust}{state_set.rs}
	\caption[Rust implementation of an arbitrary state set.]{Definition of type \code{StateSet}, which acts as an anonymous sum type by encoding its variants as a tail-recursive tuple in its generic argument. Two non-overlapping definitions of trait \code{StateSetMatch} are provided to make the type behave as expected in response to associated method~\code{match\_head}. Function \code{example} demonstrates how the arbitrary number of variants are matched two at a time by repeatedly attempting to match the first element of the list (the head), translating it into a conventional \code{Result} enum which Rust can pattern-match as usual. The result of this match can depend on the contents of field \code{data}, which is instantiated dynamically at runtime by interacting with the coordinator.}
	\label{listing:state_set}
\end{listing}


% representing an indeterminate state-type which will \textit{later} be chosen from the elements of the set. This type is entirely opaque other than a function to code{determine} it. At runtime, \textit{determine} blocks until the governor receives a message from the protocol, communicating a particular choice. For cases where the \code{StateSet} is a singleton, \code{determine} simply unwraps the element. While the decision is made at \textit{runtime}, the type-state automaton exists at runtime. To make this possible, the programmer must provide behavior for each \textit{case}, corresponding to elements of the set, one of which will be chosen by \code{determine}. This use case describes a \textit{sum type}, which is already present in rust as the \code{enum}, a union type with a set of \textit{variants}. However, every new variant set would necessitate the definition of a new enum. This is impractical, as the number of combinations are large. Instead, we implement our own types which behave and appear to the user much like \textit{anonymous sum types} (which do not currently exist in the rust language). This is achieved by relying once again on Rust's trait system to encode \textit{lists} in type-space, using nested tuples. \textit{Matching} of these nested tuple types translates to peeling away tuple layers.

\subsection{Governed Environment}
\hl{TODO}

\section{Making it Practical}
With a basic outline for the implementation, we are able to realize some functional, yet na\"ive governors. However, there is a long way to go before these systems can be applied in any realistic scenario. In this section, we explain which problems remain to be solved, whether for the sake of managing complexity, or for the user's ergonomics.

\subsection{Approximating the RBA}
The approach to generating a type-state automaton from an RBA was given in~\ref{sec:type_space_automaton}. Our type-state automaton suffer the same state-space explosion of Constraint Automata, prior to the inclusion of memory (explained in Section~\ref{sec:semantic_models}). We cannot hope to represent realistic programs with this approach alone, as the type-state automata would be wildly unmanageable in its number of states and transitions. In this section, we explain how the type-state automaton is adapted to \textit{approximate} the protocol's configuration space such that we strike a balance between accuracy and simplicity, without any effect on the governor's correctness.

\subsubsection{Data Domain Collapse}
% making RBA automata feasible
\label{sec:approximating_rba}
We abandon the goal of faithfully representing the entirety of the protocol's configuration space in favor of representing an approximation by assuming all data types to be the trivial \textit{unit-type}. With this assumption, memory cells may be in one of two states: (a) empty, (b) filled with `unit'. Converting existing RBAs may see large sub-expressions of \textit{data constraints} becoming constant, including checks for equality and inequality between port values. This assumption is justified by its relation to Assumption~\ref{as:1} from Section~\ref{sec:minimizing_bottleneck}. In this context, it can be understood to mean that \textit{usually}, two configurations that are only distinguished by having different \textit{data values} in memory cells or begin put by putters satisfy precisely the same subset of the RBA's guards. Consequently, they do not need to be distinguished. This simplification greatly reduces the total number of types to encode an RBA's configuration space. However, it is still necessary to consider the possible \textit{combinations} of all empty and full memory cells, requiring potentially $2^N$ types for $N$ cells. Rather than enumerating these types explicitly, we can rely on the \textit{structure} the RBA provides by simply encoding each automaton configuration as a \textit{tuple} of types \code{Empty} and \code{Full}. In a sense, each tuple is indeed its own type, but neither the code generator nor the compiler need to pay the price of enumerating all the combinations eagerly. For example, a configuration of three empty memory cells would be represented by type \code{(Empty,Empty,Empty)}. For brevity, we will henceforth abbreviate these tuples by omitting commas, and shortening~\code{Empty} and~\code{Full} to~\code{E} and~\code{F} respectively.


As before, we are able to represent an RBA rule as a \textit{function} in the Rust language by encoding a configuration change from $q$ to $p$ determines its \textit{declaration} such that it consumes the type-state of $p$ and returns the type-state of $q$. The na{\"i}ve approach of generating functions per type-state is susceptible to the same \textit{exponential explosion} that plagued CAs in the first place. Fortunately, tuple-types have inherent structure which Rust's generic type constraints are able to understand. The use of generics to \textit{ignore} elements of the tuple coincides with an RBA's ability not \textit{ignore} memory values. Consequently only one function definition per RBA rule is required. The way the rule's data constraint manifests is somewhat different, as our function must \textit{explicitly} separate the \textit{guard} and \textit{assignment} parts and represent them as constraints on the parameter-type and return-type respectively. As an example, Listing~\ref{listing:fifo_tsa} demonstrates the type definitions and rule functions for the \textit{fifo2} protocol first seen in Section~\ref{sec:semantic_models} with the associated RBA shown in Figure~\ref{fig:fifo2_rba}. Observe that the concrete choices for tuple elements act as \textit{value checks} for memory cells in either empty or full states. Omission of a check must be done explicitly using a \textit{type parameter} such that the function is applicable for either case of \code{Empty} or \code{Full}, and to ensure the \textit{new} state preserves that tuple element; this causes memory cells to have the expected behavior of \textit{propagating} their values into the future unless otherwise overwritten by assignments. This serves as an example of a case where our simplification coincides with a faithful encoding of the original protocol as \textit{fifo2} never discriminates elements of the data domains of~$A$ and~$B$.


\begin{listing}[ht]
	\inputminted[]{rust}{fifo_tsa.rs}
	\caption[Type state automaton in Rust for the fifo2 connector.]{Type-state automaton for the \textit{fifo2} protocol in Rust. The three latter functions correspond to the three rules seen for the RBA in Listing~\ref{fig:fifo2_rba}. Function bodies are omitted for brevity. Note that \code{M} is not a type, but rather a generic \textit{type parameter} to be instantiated at the call site.}
	\label{listing:fifo_tsa}
\end{listing}



%For the sake of brevity in sections to come, we introduce a more compact syntax for representing \textit{simplified} RBAs as a list of rules, each of which is:
%
%\begin{tabular}{r|rcll}
%	$index$ & $guard$ & $\rightarrow$ & $assignment$ & $ports$ \\
%\end{tabular}
%\\
%where:
%\begin{itemize}
%	\item $index$ is a numeric index to facilitate referring to specific rules in text
%	\item $guard$ is a string of concatenated symbols $g_0g_1g_2 ...$ from alphabet set \{\texttt{E},\texttt{F},\texttt{.}\} such that $g_i$ \textit{guards} the value of the $i$th memory cell. \texttt{.} is \textit{arbitrary} and thus the memory cell is not checked.
%	\item $assignment$ is formatted similarly to $guard$, but determines the \textit{assignment} of memory cells in the next time stamp. Here, \texttt{.} \textit{carries over} the value from the guard for that memory cell.
%	\item $port$ 
%\end{itemize}




\subsubsection{RBA Projection}
% removing silent transitions p1
\label{sec:rba_projection}
When a protocol's interface is provided as-is to a compute component, its model itself (an RBA in our case) defines precisely what it is permitted to do, just with the \textit{direction} of operations reversed; for the component to be compatible, it must put on port $P$ whenever the protocol gets on $P$, and get on port $Q$ whenever the protocol puts on port $Q$. In such cases, the procedure for encoding the RBA described in Section~\ref{sec:approximating_rba} can be applied directly. Otherwise, the interface of a compute component does not subsume the entirety of the interface of its protocol. In such systems, the protocol interfaces with several compute components. Indeed such cases form the majority in practice; compute components tend to only play a small role in a larger system.

The contents of Section~\ref{sec:approximating_rba} are sufficient to generate some functional governors. We consider a system containing protocol $P$ and connected compute component $C$ with interfaces (port sets) $I_P$ and $I_C$ respectively such that $I_P \supseteq{} I_C$. We wish to generate governor $G_C$ whose task is to ensure that $C$ adheres to $P$. As a first attempt, we translate $P$'s RBA to Rust functions and types as-is. We would quickly notice that the RBA's data constraints represent port-operations that are excluded from $I_C$. These interactions involve no actions on $C$'s part; from the perspectives of $C$ and $G_C$, these actions are \textit{silent}. Equivalently, we do not use the RBA of $P$ directly, but consider instead its \textit{projection} onto $I_C$, which \textit{hides} all actions that are not in the interface projected upon. 


\begin{table}
	\centering
	\begin{tabular}{l|ll|}
		rule & guard & assignment \\
		\hline
		0 & $m_0=*$ & $\gapwedge{}m_0'=d_A$\\
		1 & $m_0\neq{}*\gapwedge{}m_1=*$ & $\gapwedge{}m_1'=m_0\gapwedge{}m_0'=*$ \\
		2 & $m_1\neq{}*$ & $\gapwedge{}d_B=m_1\gapwedge{}m_1'=*$ \\
		\hline
	\end{tabular}
	\caption[RBF for the fifo2 connector.]{RBF of the \textit{fifo2} protocol, equivalent to the RBA in Figure~\ref{fig:fifo2_rba}. Formatted with an outermost disjunct per line such that guard and assignment parts per rule are discernible.}
	\label{tab:fifo2_rbf_tsa}
\end{table}

\begin{listing}[ht]
	\inputminted[]{rust}{fifo_tsa_2.rs}
	\caption[Type state automaton in Rust with silent rules.]{Type-state automaton rules which govern the behavior of a compute component with interface ports $\{A\}$ for the \textit{fifo2} protocol. Function bodies list the \textit{actions} which the component contributes to the system. Observe that rules but 0 are silent.}
	\label{listing:fifo_tsa2}
\end{listing}

As an example, we once again generate a governor for a compute-component with interface $\{A\}$ with the \textit{fifo2} protocol. This time the protocol is represented as an RBF in Table~\ref{tab:fifo2_rbf_tsa} to make the correspondence to the generated governor in Figure~\ref{listing:fifo_tsa2} more apparent. Observe that all but one of its rule functions are \textit{silent}, serving no purpose but to advance the state of the automaton by consuming one type-state and producing the next. As demonstrated here, this approach to generating governors is correct, but has two undesirable properties:
\begin{enumerate}
	\item \textbf{API-clutter}\\
	The end-user is obliged to invoke functions which correspond with rules in the protocol's RBA. In many cases, these rules will serve no purpose other than to consume a type-state parameter, and return its successor.
	
	\item \textbf{Protocol Entanglement}\\
	The type-state automaton captures the structure and rules of the protocol's RBA in great detail. This is a failure to \textit{separate concerns}, which further couples the compute component to its protocol. This has the immediate effect of making components difficult to re-use (their implementations are more protocol-specific), as well as making them brittle to \textit{changes} to the protocol, making them difficult to maintain. 
	
\end{enumerate}

\subsubsection{RBA Normalization}
% removing silent transitions pt 2
\label{sec:rba_normalization}
Section~\ref{sec:rba_projection} introduced a procedure for generating governors, but also discussed a significant weakness; all governors are represented by type-state automata based on the original protocol's rules. In this section, we introduce a notion of \textit{normalization} that intends to \textit{specialize} the governors according to its needs such that it is still `compatible' with the protocol's RBA in all ways that matter, but has greatly reduced \textit{api-clutter} and \textit{protocol-entanglement}. 

Let an RBA be in normal form if it has no silent rules. We observe that the presence of silent rules contributes to both api-clutter and protocol entanglement. Ideally, we wish to abstract away the workings of the protocol as much as possible; at all times, the governor only needs to know which actions the component must perform \textit{next}. To make this notion more concrete, we introduce some definitions which build on one another to define the term we need: our normalization procedure should generate an RBA with starting configuration which \textit{port-simulates} the protocol's RBA in its starting configuration:
\begin{itemize}
	\item $Act(r)$ of an RBA state $r$:\\
	The set of ports in $r$ which perform actions (ie: are involved in interactions).
	
	\item \textit{Rule sequence} from $c_0$ to $c_1$ of RBA $R$:\\
	Any sequence of rules in $R$ that can be applied sequentially, starting from configuration $c_0$ and ending in configuration $c_1$.
	
	\item \textit{$P$-final} wrt.\ port set $I$:\\
	A rule sequence of RBA $R$, with \textit{last} rule $r_{last}$ is $P$-final with respect to port set $I$ if $Act(r_{last})\cap{}I=\{P\}$ and for all rules $r$ in the sequence, $r=r_{last} \lor{} Act(r)\cap{}I=\varnothing$.	
	
	\item RBA $R_1$ in config.\ $c_1$ \textit{port-simulates} $R_2$ in config.\ $c_2$ wrt.\ Interface $I$:\\
	If for every $P$-final rule sequence of $R_2$ starting in $c_2$, ending in $c_2'$ there exists some $P$-final rule sequence of $R_1$ starting in $c_1$, ending in $c_1'$ such that $R_1$ in $c_1'$ port-simulates $R_2$ in $c_2'$.
\end{itemize}

The intuition here is that it does not matter how the governor's RBA structures its rules. It is unnecessary for governors to advance in lockstep with the protocol to the extent that they agree on the protocol's \textit{configuration} at all times. It suffices if the protocol and governor always agree on which \textit{actions} the ports in their shared interface do next. Figure~\ref{fig:path_sim} visualizes this idea; observe how the normalized RBA has entirely different transitions (different labels and configurations), but is ultimately able to pair actions of the protocol for ports in its interface with its own local actions.

\begin{figure}[ht]
	\centering
	\footnotesize
	\begin{tikzpicture}[node distance=2cm, inner sep=0.5mm ]
	
	\node[state, initial, initial text=(a)\\\\] (qee) {\code{(E,E)}};
	\node[state, right of=qee] (qfe) {\code{(F,E)}};
	\node[state, right of=qfe] (qef) {\code{(E,F)}};
	\node[state, right of=qef] (qee2) {\code{(E,E)}};
	\node[state, right of=qee2] (qfe2) {\code{(F,E)}};
	
	\node[state, initial, below of=qee, yshift=7mm, initial text=(b)\\\\] (qee') {\code{(E,E)}};
	\node[state, right of=qee'] (qfe') {\code{(F,E)}};
	\node[state, right of=qfe'] (qef') {\code{(E,F)}};
	\node[state, right of=qef'] (qee2') {\code{(E,E)}};
	\node[state, right of=qee2'] (qfe2') {\code{(F,E)}};
	
	\node[state, initial, below of=qee', yshift=7mm, initial text=(c)\\\\] (qee'') {\code{(E,E)}};
	\node[state, below of=qfe', yshift=7mm] (qfe'') {\code{(F,E)}};
	\node[state, below of=qfe2', yshift=7mm] (qfe2'') {\code{(F,E)}};
	
	\draw
	(qee) edge[above] node{\code{A.get}} (qfe)
	(qfe) edge[above] node{$\cdot$} (qef)
	(qef) edge[above] node{\code{B.put}} (qee2)
	(qee2) edge[above] node{\code{A.get}} (qfe2)
	
	(qee') edge[above] node{\code{A.put}} (qfe')
	(qfe') edge[above] node{$\cdot$} (qef')
	(qef') edge[above] node{$\cdot$} (qee2')
	(qee2') edge[above] node{\code{A.put}} (qfe2')
	
	
	(qee'') edge[above] node{\code{A.put}} (qfe'')
	(qfe'') edge[above] node[pos=0.9]{\code{A.put}} (qfe2'')
	;
	\end{tikzpicture}
	\caption[RBAs in lockstep with and without normalization.]{Rules being applied to walk three RBAs in lockstep, with time horizontally, showing the (simplified) configurations traversed, and annotating rules by showing which port actions they involve.\\(a) RBA of protocol \textit{fifo2}. (b) RBA of \textit{fifo2} projected onto port set $\{A\}$. (c) RBA of \textit{fifo2} projected onto port set $\{A\}$ and normalized to remove silent rules.}
	\label{fig:path_sim}
\end{figure}



\begin{listing}[ht]
	\inputminted[]{rust}{normalize.rs}
	\caption[Normalization procedure Rusty-pseudocode.]{Normalization procedure, expressed in (simplified) Rust code. In a nutshell: while one exists, an arbitrary silent rule $x$ is removed, and the list of rules is extended with composed rules $x\cdot{}y$ such that $y$ is another rule.}
	\label{listing:normalize}
\end{listing}

The final normalization procedure is given in Listing~\ref{listing:normalize} in the form of simplified Rust code. It works intuitively for the most part: silent rules are removed, and new rules are added to retain their contribution of moving the RBA through configuration space. The function \code{normalize} ensures that the returned rule set is in the same configuration as the protocol after matching a non-silent, but the configuration is allowed to `lag behind' while the protocol performs rules which it considers to be silent. New rules must be added to `catch up' to the protocol after any such sequence of silent rules. The procedure does this by building these \textit{composed} rules from front to back, ie. replacing every silent rule $x$ with a \textit{set} of rules $x\cdot{}y$, where $y$ is any other rule. Once completed, the RBA may contain rules that are subject to \textit{simplification}. For example, \{$m=*\wedge{}n=*$, $m\neq{}*\wedge{}n=*$\} can be represented by only $n=*$.

The normalization algorithm is \textbf{correct} as clearly it does not have silent rules once it returns (\code{not\_silent} containing zero silent rules is invariant). Observe that for each silent rule removed, it does not consider composing with \textit{itself}. The immediate result is that the algorithm never inserts some rule $x\cdot{}x$ for silent rule $x$. This is not a problem, as all \textit{silent} rules of our approximated RBAs are \textit{idempotent} with respect to their impact on the configuration. The algorithm is able to take for granted that the result any \textit{chain} of silent rules $x\cdot{}x\cdot{}x\cdot{}...$ is covered by considering $x$ itself. Furthermore, the incremental removal of rules prohibits the creation of any silent cycles at all. This is due to the reasoning above being extended to any sequences also. (TODO PUMPING LEMMA).


The normalization algorithm is \textbf{terminating}. It consists of finitely many \textit{algorithm steps} in which the RBA $A$ is replaced by RBA $B=(A \setminus{}\{r\}) \cup{} \{r\cdot{}x | r\in{} A\setminus{}\{x\} \wedge{} composable(x,r)\}$ for some silent rule $x \in{} A$. Initially, $A$ is the input RBA with silent rules. The algorithm terminates, returning $B$ when $A$ is replaced by $B$ where $B$ has no silent rules. Let $P(x)$ be the set of \textit{acyclic paths} through RBA $x$'s configuration space. Observe that initially, $P(A)$ is finite. It suffices to show that in each algorithm round, $|P(A)|$ strictly decreases. Within a round, for every `added' $p$ in $P(B)\setminus{}P(A)$, $p$ contains a rule $m\cdot{}n$ such that there exists $p'$ in $P(A)\setminus{}P(B)$ identical to $p$ but with a 2-long sequence of rules $m, n$ in the place of $x$. From this we know that $|P(A)| \geq |P(B)|$. However, the 1-long path of $x$ itself is clearly in $P(A)\setminus{} P(B)$. Thus, $|P(A)| > |P(B)|$. \textsc{qed}.

To demonstrate the normalization procedure, Table~\ref{tab:fifo2_rbf_tsa_norm} shows the result of projecting the \textit{fifo2} connector's RBF onto port set $\{A,B\}$ and normalizing. The two additional rules can be understood to `cover' the behavior lost as a result of omitting the silent rule 1 from the original~RBF.


\begin{table}
	\centering
	\begin{tabular}{l|ll|}
		rule & guard & assignment \\
		\hline
		0 & $m_0=*$ & $\gapwedge{}m_0'=d_A$\\
		2 & $m_1\neq{}*$ & $\gapwedge{}d_B=m_1\gapwedge{}m_1'=*$ \\
		\hline
		$1\cdot{}0$ & $m_0\neq{}*\wedge{}m_1=*$ & $\gapwedge{}m_0'=d_A\wedge{}m_1'=m_0$ \\
		$1\cdot{}2$ & $m_0\neq{}*\wedge{}m_1=*$ & $\gapwedge{}m_0'=*$ \\
		\hline
	\end{tabular}
	\caption[RBF of fifo2 connector, projected and normalized.]{RBF of the \textit{fifo2} protocol, projected onto port set $\{A,B\}$ and normalized. Rules 0 and 2 are retained from Table~\ref{tab:fifo2_rbf_tsa}, and new rules $1\cdot{}0$ and $1\cdot{}2$ are composed of rules from the original RBF.}
	\label{tab:fifo2_rbf_tsa_norm}
\end{table}


\subsection{User-Defined Protocol Simplification}
\label{sec:user_defined_simplification}
Recall, the purpose of a governor is to preserve a system's \textit{liveness}.
They do this by ensuring that their governed compute component performs port operations that allow the interfacing protocol (and the system around it) to progress. 
Governors do this by enforcing that their compute component's implementation \textit{covers} each possible transition with code that performs the required task, and ensuring it is chosen correctly in accordance with the wishes of the protocol. Section~\ref{sec:rule_consensus} explains how our type-state automaton represents this by each configuration requiring the definition of a \textit{set} of transitions, one for each action. We say the implementation `covers' each of these cases by defining the component's behavior in each cases, including the invocation of the relevant port operation.

An overzealous governor which requires to cover \textit{additional} (unnecessary) cases would still serve its purpose. In effect, such a governor would enforce adherence to some \textit{other}, more permissive protocol. However, liberty of the protocol means responsibility to the compute-component: the more the protocol \textit{might do}, the more the compute-component \textit{must consider doing}. There is incentive for governors to do this: permissive protocols are simpler to enforce.

This conservatism becomes a problem when it infringes on the component's ability to express its behavior as intended. Consider the example of a compute-component~$X$ that forwards values from its input port~$A$ to its output port~$B$. Perhaps this component is used in a pipeline as intended such that the component is involved in an endlessly-alternating sequence, represented by regular expression~$(AB)^*$. Perhaps there is a sensible way for $X$ to implement the more permissive protocol which permits~$B$ firings to be omitted, expressed $(A(B|\lambda{}))^*$. $P$ has no problem discarding values input from $A$. However, if the governor takes it a step further such that `anything goes' (expressed $(A|B)^*$), $X$~cannot meaningfully represent its work. How on earth can it forward a message to $B$ before receiving it on~$A$? Not even clairvoyance can help; what if $A$ never fires at all? This is how the user would experience the problem of a governor infringing on the component's own behavior; in a sense,~$P$ has a protocol of its own which must be preserved on its interface ports which the governor violates.



\begin{table}
	\centering
	\begin{tabular}{l|ll}
		rule & guard & assignment \\
		\hline
		0 & $m_0=*$
		& $\gapwedge{}m_0'=d_A$\\
		1 & $m_0\neq{}*\wedge{}m_1=*$
		& $\gapwedge{}m_1'=d_A\wedge{}m_0'=*$\\
		2 & $m_0=m_1\neq{}*\wedge{}m_2=*$
		& $\gapwedge{}m_2'=d_A\wedge{}m_0'=m_1'=*$\\
		3 & $m_0=m_1=m_2\neq{}*$
		& $\gapwedge{}d_B=m_0\wedge{}m_0'=m_1'=m_2'=*$\\
		\hline
	\end{tabular}
	\caption{RBF of the \textit{a7b1} connector, which is characterized by cycling through a predictable sequence of period~8, where~$A$ inputs seven times and~$B$ outputs once. It works by encoding its configuration in an 8-long cycle as a three-bit integer using the fullness of memory cells $m_{0-2}$.}
	\label{tab:counting_rbf}
\end{table}

Nevertheless, there is value in providing a compute-component with a simplified (permissive) view of the protocol where possible. As a motivating example, consider the \textit{a7b1} connector, given as RBF in Table~\ref{tab:counting_rbf}. This connector uses the fullness of three memory cells to count in binary from zero to seven (using the binary alphabet of memory cell states $\{\text{\code{E}}, \text{\code{F}}\}$), and cycling back again to zero. Configurations in this cycle are distinguished by specifying different behaviors on~$A$ and~$B$. Here, the projection and normalization of the protocol's RBF is trivial, as no rules are silent. Without the ability to simplify, the~$Y$ must be implemented such that it corresponds exactly with the protocol's (predictable) walk through its approximated configuration space, given in Figure~\ref{fig:counter_RBAs}. As all states are distinguishable, so too are their corresponding \textit{state types} distinct. Now consider this protocol interfacing with some compute-component~$Y$, which is always ready to consume and emit some data element~\textit{Q}. Without simplification, the resulting governor would require that the traversal through configuration space be spelled out; the user would be forced to distinguish these states, even though~$Y$ has no need for this specificity. Most likely, the resulting implementation will be repetitive and verbose, if the behavior is the same for configurations \code{(EEE)}, \code{(EEF)}, et cetera.

\begin{figure}[ht]
	\centering
	\footnotesize
	\begin{tikzpicture}
	[ inner sep=0.6mm ]
	\matrix (m) [ matrix of nodes, row sep=0.5cm, column sep = 0.7cm, 
	nodes = {anchor=center,circle, draw=black, thick, fill=gray!10},] {
		\node(m0)[initial]{\texttt{EEE}}; & \node(m1){\texttt{EEF}}; & \node(m2){\texttt{EFE}}; & \node(m3){\texttt{EFF}}; & \node(m4){\texttt{FEE}}; & \node(m5){\texttt{FEF}}; & \node(m6){\texttt{FFE}}; & \node(m7){\texttt{FFF}};\\
	};
	\draw
	(m0) edge[above] node{$A$} (m1)
	(m1) edge[above] node{$A$} (m2)
	(m2) edge[above] node{$A$} (m3)
	(m3) edge[above] node{$A$} (m4)
	(m4) edge[above] node{$A$} (m5)
	(m5) edge[above] node{$A$} (m6)
	(m6) edge[above] node{$A$} (m7)
	(m7) edge[bend left, above] node{$B$} (m0)
	;
	\end{tikzpicture}
	\caption[Configuration space of the a7b1 connector.]{Rules transitioning through configuration space of approximated RBAs for \textit{a7b1} connector, with states named after the `count' the three memory cells represent in base~2 (in binary alphabet $\{\text{\code{E}}, \text{\code{F}}\}$). Here, the normalization procedure with interface set~$\{A,B\}$ is trivial as no transitions are silent.}
	\label{fig:counter_RBAs}
\end{figure}

Our solution to this problem is to introduce a third type for representing the state of a memory cell which may be \textit{either} full or empty:~\code{Unknown} (abbreviated as~\code{U}). Rather than corresponding to a specific configuration of the (approximated) RBA, the governor now reasons about the \textit{set} of states which the protocol may be in. For example, type \code{(UUE)} encapsulates all the concrete configuration types $\{\text{\code{(EEE)}}, \text{\code{(EFE)}}, \text{\code{(FEE)}}, \text{\code{(FFE)}}\}$, and is liable to covering the \textit{union} of the rules applicable to any of those states. In this manner, it is safe for the programmer to arbitrarily `forget' the state of a memory cell, replacing its element in the tuple type with~\code{U}. To be clear, \code{U} is not special as far as Rust is concerned; we have changed to a ternary alphabet for representing memory cells in types. However, \code{U} does not correspond to any real configuration that memory cells are ever `really' in at runtime; they are always either empty or full. \code{U} is a stand-in for an empty \textit{or} full memory variable, an abstraction in which the protocol is not (explicitly) involved.
With this tool in their belt, the implementation of the compute component is able to arbitrarily \textit{unify} the state-types of multiple branches. Our example component~$Y$ above is able to implement its behavior to the satisfaction of its governor with transitions through configuration space in Figure~\ref{fig:weakening}. This weakening can be communicated quite ergonomically, resulting in something very close to what the user would implement themselves: a single loop where the four rules (numbered 0-3) may be applied to configuration type~\code{(UUU)}, each resulting again in~\code{(UUU)}.


\begin{figure}[ht]
	\centering
	\begin{tikzpicture}[node distance=2cm, inner sep=0.5mm ]
	\node[state, initial]      (q000) {\texttt{EEE}};
	\node[state, below right of=q000] (q???) {\texttt{\textcolor{red}{UUU}}};
	\node[state, above right of=q???] (q??1) {\texttt{\textcolor{red}{UU}F}};
	\node[state, below right of=q???] (q?10) {\texttt{\textcolor{red}{U}FE}};
	\node[state, below left of=q???] (q100) {\texttt{FEE}};
	\draw
%	weakenings
	(q000) edge[dashed, bend left] node{} (q???)
	(q??1) edge[dashed, bend left] node{} (q???)
	(q?10) edge[dashed, bend left] node{} (q???)
	(q100) edge[dashed, bend left] node{} (q???)

%	actual transitions
	(q???) edge[above left, bend left] node{0} (q??1)
	(q???) edge[above right, bend left] node{1} (q?10)
	(q???) edge[below right, bend left] node{2} (q100)
	(q???) edge[below left, bend left] node{3} (q000)
	;
	\end{tikzpicture}
	\caption[Traversed configuration space for a7b1 connector using weakening.]{Rules transitioning between configurations of the \textit{a7b1} connector shown in Figure~\ref{fig:counter_RBAs}. Here, the user employs \textit{weakening} to convert (dashed arrows) state tokens to configurations to the configuration-set `???' containing \textit{all} concrete configurations. RBA rules firing are shown with solid arrows, annotated with the rule name, corresponding to those given in Table~\ref{tab:counting_rbf}.}
	\label{fig:weakening}
\end{figure}



%How can the governor know which distinguishable states the compute components does not wish to distinguish? Our solution is to provide a means for \textit{unifying} 
%We cannot know a priori which distinguishable states the compute component \textit{wishes} to distinguish. Ideally, the user has an ergonomic means of making this choice themselves. Our solution introduces a means for 
%
%
%We currently consider memory variables of the binary domain, ie.\ each is either in state~\code{Full} or~\code{Empty}. We introduce a third state with type \code{Unknown}. Conceptually, configurations with \code{Unknown}-state memory cells represent protocol configurations where each is either concretely full or empty, ie.\ the governor `cannot remember'. 
%
%
%Section~\ref{sec:rule_consensus} describes how the Governor must already be prepared to handle configurations for which there are multiple possible \textit{next} transitions. By adding 
%
%As far as the compiler is concerned, \code{Unknown} is just another state, as distinct from \code{Empty} or \code{Full} as any other. Conceptually, we use \code{Unknown} to represent th
%
%type for a new state memory cells can be in to supplement existing types \code{Empty} and~\code{Full}:~\code{Unknown}. A type-state with memory cells in an \textit{unknown} state is obliged to cover the \textit{union} of all its concrete possibilities. For example, \code{(Empty, Unknown)} may result in any states either \code{(Empty, Empty)} or \code{(Empty, Full)} may result in. With this definition, it is perfectly correct for a state-type to be `weakened', converting it into a new state-type, identical but with any of its tuple-elemets replaced by \code{Unknown}. In a sense, this approach has type-states no longer corresponding with configurations 1-to-1, but rather, each type-state encodes a simple predicate such that it represents a \textit{set} of configurations. For example, \code{(Unknown, Unknown)} represents \textit{any} configuration (of an RBA with two memory cells). Conceptually, this change has no influence on the way a governor computes which rules are applicable next, but the programmer is empowered with the ability to trade \textit{specificity} for \textit{brevity} in their compute-component implementations. Figure~\ref{fig:weakening} shows an example of how the programmer may choose to traverse their configuration space for the same connector as in the example above. In this case, using \textit{weakening} does not reduce the number of type-states encountered in the implementation; this is only the result of (a) the original RBA not being very complicated, and (b) the choice to only \textit{partially} weaken some states (It is also possible to weaken everything to \code{(Unknown, Unknown, Unknown)} and then match all rules).
%
%
%\begin{figure}[ht]
%	\centering
%	\begin{tikzpicture}[node distance=2cm, inner sep=0.5mm ]
%	\node[state, initial]      (q000) {\texttt{000}};
%	\node[state, right of=qee] (q??0) {\texttt{\textcolor{red}{??}0}};
%	\node[state, right of=q??0] (q??1) {\texttt{\textcolor{red}{??}1}};
%	\node[state, below of=q??1] (q?11) {\texttt{\textcolor{red}{?}11}};
%	\node[state, right of=q??1] (q101) {\texttt{101}};
%	\draw
%	(q000) edge[dashed] node{} (q??0)
%	(q??0) edge[above] node{0} (q??1)
%	(q??1) edge[below, bend left] node{3} (q000)
%	
%	(q??1) edge[left, bend right] node{$1\cdot{}0$} (q?11)
%	(q?11) edge[dashed] node{} (q??1)
%	
%	(q??1) edge[below, bend right] node{$2\cdot{}0$} (q101)
%	(q101) edge[dashed] node{} (q??1)
%	;
%	\end{tikzpicture}
%	\caption{Rules transitioning between configurations of the \textit{8-count cycle} connector in Figure~\ref{fig:counter_RBAs}. Here, the user employs \textit{weakening} to convert (dashed arrows) state tokens to configurations with \textit{unknown} memory cell states. RBA rules firing are shown with solid arrows, annotated with the rule name, corresponding to those given in Table~\ref{tab:counting_rbf}. Note that not all possible weakenings are shown, only an example set that covers all rules.}
%	\label{fig:weakening}
%\end{figure}


\subsection{Match Syntax Sugar}
Section~\ref{sec:rule_consensus} explains how the set of transitions to be covered by a configuration type can be represented in Rust's type system as a tail-recursive list. This alleviates the problem of having to explicitly enumerate the needed sets each as their own enumeration type. This is necessary, as the upper-bound\footnote{Many factors reduce this number drastically in practice. For example, state-sets are usually not large because they are only ever encountered when reached by \textit{transitions} from some state.} of state sets is $2^{3^M}$, where $M$ is the number of memory cells\footnote{The number of unique state sets is $2^S$, where $S$ is the number of configurations (automaton state types). This, in turn is $3^M$, as each memory cell's state is represented by a type in $\{\text{\code{E}}, \text{\code{F}}, \text{\code{U}}\}.$}; suffice it to say, it is \textit{a~lot}.
Unfortunately, these are not natively-supported enumeration types, and thus cannot be \textit{matched} as is idiomatic in the Rust language. However, Rust has extensive support for abstract syntax tree \textit{macros}, allowing us to have the best of both worlds; the user interacts with \code{StateSet} types by using a match-like macro which enumerates the branches, but there is no need for concrete \code{enum} classes to be defined for all the conceivable combinations. Figure~\ref{listing:sugar} gives an example of how these cases compare to one another.


\begin{listing}[ht]
	\centering
	\inputminted{rust}{sugar.rs}
	\caption[Example of mimicing enum matching with a sugaring macro.]{Example of three methods for matching a \textit{state set}, representing a sum type of three variant types simplified here to \code{X}, \code{Y} and \code{Z}. First, \code{match\_standard} shows how this is done in idiomatic Rust, requiring an enum type \code{StateSetXyz} be explicitly defined. \code{match\_recursive} shows how the same state set represented by a tail-recursive \code{StateSet} type can be similarly matched by exhaustively `unzipping' head elements using a function \code{match\_head}. finally, \code{match\_macro} functions identically to the second case, but relies on a sugaring macro \code{match\_set} to mimic the syntax of Rust's \code{match} statement, seen in the first function.}
	\label{listing:sugar}
\end{listing}