\chapter{Protocol Translation}
\label{sec:imperative_form}
In this chapter, we describe the process of translating the Reo compiler's internal representation of a protocol specification into an executable \textit{protocol object} in the Rust language.

\section{Two-Phase Code Generation}
\label{sec:two_phase}
In this section, we explain and motivate our approach of segmenting the code generation process into two distinct phases. Throughout, we refer to the precedent set by the existing Reo compiler backend for generating code in the Java language, as it has seen the language most similar to Rust which has seen significant development.   

\subsection{Generation Sub-tasks}
\label{sec:sub_tasks}
Reo specifications represent connectors declaratively as relations between ports. They are thus well-suited to reasoning about the protocol's properties. In contrast, our target imperative languages such as Java and Rust represent computation such that it corresponds more closely to machine instructions; they are imperative, laying out sequences of actions which together emerge as interaction at runtime. Where interactions in the former can be oriented around the synchronous observations of port values, interactions of the latter must be expressed as sequences of actions, laid out over time. Implementing algorithms for translating between these forms must take care that the translation procedure between these forms preserves the semantics as intended; choosing the incorrect ordering can change the nature of the emergent interaction in unexpected ways. For example, reading memory cell \textit{before} writing it corresponds to a different interaction than reading it \textit{after} writing it. 

Java, Rust and Reo have in common that they are strongly-typed languages. Reo's specifications are permitted a degree of \textit{type elision}; for the sake of programmer ergonomics, the data-types of ports may be omitted, such that they can later be derived in context. Rust shares this property, and so the Rust compiler works to \textit{resolve} data types during compilation. Circumstantially, these elisions may produce cases for which a correct resolution is impossible, as the type annotations or constraints present are in conflict. Our task is to emulate this work ourselves, assigning concrete types for port objects in our emitted code such that is guaranteed to type-check in the Rust language. Failure to do this correctly would result in Reo emitting code rejected by the Rust compiler. This would not be a threat to correctness, but it results in significant inconvenience to the programmer.

Regardless of the intermediate representation, protocol objects must ultimately be emitted in the target language. Aside from expression in the correct syntax, the end result must make explicit any work required to make it \textit{executable} with the desired runtime behavior. Even simple concepts require the support of auxiliary book-keeping structures to maintain the protocol object's state, and specialized \textit{concurrency primitives} are needed to ensure that actions compose into interactions at runtime in the expected way. Clearly, this is very particular to the target language, as they vary greatly on how they fundamentally express operations on data at a granular level.

In summary, we identify and name three sub-tasks of generating target language protcol objects from a Reo protocol description:
\begin{itemize}
	\item [$T_{seq}$] Declarative interactions must be decomposed into sequences of imperative actions.
	\item [$T_{type}$] Ports must be given data types such that they agree with any type-annotations in the Reo specification, and successfully type-check in the target language.
	\item [$T_{run}$] Details necessary to make the result runnable are included. Symbolic actions are represented as concrete operations on data.
\end{itemize}


\subsection{Decoupling the Reo Compiler from Rust}
\label{sec:decoupling_reo_rust}
The Reo compiler has an existing backend for generating Java code. It works by generating Java according to the structure of a \textit{template generator}. In this manner, it can be thought of as performing all code-generation sub-tasks at the same time directly from the compiler's intermediate representation. However, the extent to which the Reo compiler is \textit{coupled} to the Java language is reduced through the reliance on a Java library for the granular implementations of structures that are common to all protocol objects; rather than generating these classes each time, the Reo compiler simply generates a dependency. For example, the library defines a \code{Component} interface, for which the code generator produces a protocol-specific implementor class. Consequently, a significant part of the $T_{run}$ sub-task (sub-tasks are defined in Section~\ref{sec:sub_tasks}) is delegated to this library.

For $T_{type}$, help comes from the Reo compiler itself, which in its current form was developed with support for Java in mind. This is visible in its internal representation. For example, types for which no explicit data type annotation was included are assigned the \code{Object} type, which encapsulates all types that may be concretely chosen for transmission through ports. This design essentially uses \code{Object} as an all-encompassing \textit{sum type}, relegating the task of \textit{type reflection} (determining concretely which `variant' of \code{Object}) to the user themselves. This approach is sufficient in the case of Java, as \code{Object} supports all the behavior relied about by the Reo protocol object at runtime, namely (1) data-equality checks, (2) data movement, and (3) data replication\footnote{In the chapter to follow we discuss how this approach introduces safety concerns. In a nutshell, Reo-generated Java assumes that the replication of \code{Object} references preserves Reo's value-passing semantics. This is not the case, as it may result in \textit{data races}.}.

Only $T_{seq}$ is performed almost entirely by the template generator. For simple protocols, this task is relatively easy, as there is not much to add when actions are largely concurrent. For example, replicating the contents of a memory cell into a set of others is simply-done in Java by first reading an object reference, and then overwriting the others one at a time. However, ordering dependencies must be resolved very carefully in the general case. The current Java code generator is susceptible to erroneously observing value $x$ at memory cell $M$ in the event that the observation is synchronous with $M$'s value being overwritten by $x$. Even with the help of the template generator, this translation is sufficiently complex to make detection of these bugs difficult.

Rust is able to mimic Java's approach to create a similar backend through the explicit use of \textit{dynamic dispatch}, such that types can be collapsed to something analogous to Java's \code{Object} class. If done na\"ively, the resulting backend would inherit the problems of its predecessor, and new ones to boot; the Java-like approach is not idiomatic in the context of Rust; it would not make good use of the extensive control of systems resources unique to a systems language. Chapter~\ref{sec:protocol_runtime} to follow goes into detail about the properties of the protocol runtime. Here, it suffices to say that we wish to implement a runtime that does not rely on heap-allocation of its port-values, and thus cannot rely entirely on dynamic dispatch. Furthermore, our runtime wishes to perform more extensive optimizations, relying on the unique abilities of our systems language to manipulate its resources at a low level. All these extensions pose a problem in particular for $T_{run}$, as runtime properties directly influence how the executable protocol objects are represented. Our work in unremarkable in its solution to this problem: we delegate~$T_{seq}$ to a Rust library. However, we make this separation more extreme. In a nutshell, we wish to partition the work of code generation into two clear \textit{phases}, the former of which performs tasks~$T_{seq}$ and~$T_{type}$, and the latter of which performs $T_{seq}$. To minimize coupling between the modules performing these tasks, the interface between phases is made terse, unambiguous and explicit in the definition of a new intermediate representation of protocol connectors: the \textit{imperative form}.

\subsection{Temporary Simplifications}
\label{sec:temporary_simplifications}
Our intention is to isolate the Reo backend from Rust's specifics as extensively as possible. In this manner, we decouple the modules responsible for the code-generation subtasks in accordance with good software engineering practices. Furthermore, it facilitates the \textit{re-use} of the first phase of the code generation process for \textit{other} imperative programming language targets. The section to follow defines imperative form to be as target-language agnostic as possible. However, for the sake of minimizing the disturbance to the Reo tooling ecosystem, we still embrace the per-target structure for Reo code generation for now. As such, the Reo compiler still specifies a Rust language target, and emits executable Rust source as a result. Our representation of the \textit{imperative form} is expressed in Rust syntax (as the \code{ProtoDef} type) such that this reliance on an intermediate representation is invisible to the end user. As far as they are concerned, Reo generates native Rust that just happens to \textit{somehow} make minimal use of Rust-specific syntax. \code{ProtoDef} corresponds closely to the definition of imperative form, facilitating this decision being overturned in future with minimal effort.


\section{Imperative Form}
\label{sec:imperative_form_sec}
In this section, we define our new intermediate representation of Reo protocol specifications. We include an intuitive look at how it captures the details of the Reo compiler's internal representation, but such that only $T_{seq}$ remains to be performed before the finished Rust source code can be emitted.

\subsection{Concept}
The Reo compiler's internal representation does not ergonomically facilitate execution, primarily because it does not define the \textit{order} in which values are accessed, created and moved. Programmers using imperative, sequential languages are very used to thinking in terms of procedures which manipulate the state of variables \textit{in scope} with the order implicit in their control flow. Often, interpreters or compilers provide safety properties by tracking over the execution and emitting errors whenever a variable access is invalid.

Essentially, imperative form makes explicit the ordering between symbolic \textit{actions}; if executed in the specified order, it is guaranteed that (1) variable accesses are always valid, and (2) it is clear at which moment the rule has \textit{fired}.

\subsubsection{Relationship to Reo and Target Languages}
Imperative form represents a protocol whose translation from Reo to an imperative language has been completed as fully as possible, but stopped short of introducing implementation- and language-specifics. Thus it is still a specification, free from particular syntax, and rendered in terms of \textit{symbolic} identifiers and data types to be resolved in the manner befitting the target language. In terms of the generation sub-tasks defined in Section~\ref{sec:sub_tasks}, imperative form represents the completion of~$T_{seq}$ and~$T_{type}$, but not~$T_{run}$.

The translation from Reo's internal representation to imperative form is \textit{lossless}, and so any language compiling from the former would also be able to compile from the latter. However, the utility of this representation is the increase in \textit{explicitness}, which results from the ordering of actions. This ordering follow from the fundamental assumption of imperative form: a value can only be accessed \textit{after} it has been created. It also inherits an assumption of the Reo language itself; namely, all values and their identifiers can be assigned a static data type\footnote{Imperative form assumes that the target language can assign static data types to ports. However, this assumption is shared by Reo itself, and does not present a problem for untyped languages. For imperative languages without types, ports can simply share some~\code{Any} type, satisfying this assumption trivially.}.

\subsubsection{Rules as Transactions}

The Reo compiler's internal representation partitions the work of a rule into its \textit{guard} and \textit{assignments}. This is already a step in the direction of imperative computation, observing that some work (the guard) must be performed \textit{prior} to deciding whether the rule \textit{fires}, in which case the assignment follows. As the protocol does not define the moments when it will evaluate the guard, it is necessary that this evaluation has no side effects ie.\ observable effects to the outside world. In essence, Reo's internal representation formulates a rule as a two-element sequence where the first (the guard) is \textit{transient}, and may end the rule's execution early, and the second (the assignment) exhibits \textit{observable effects}, namely, the results of the rules \textit{firing}.

Imperative form adopts this notion of ordering, but generalizes it to a sequence of arbitrary length. For our purposes, it suffices to continue requiring a single \textit{last} action to represent the assignment. For all prior actions:
\begin{enumerate}
	\item they have a defined means of being \textit{undone}, ie.\ the action must be reversible. It follows that each action cannot have any immediately-observable effects,
	\item they may conditionally trigger an \textit{abort} event, which occurs after their evaluation completes.
\end{enumerate}

Effectively, we represent each of the protocol's rules as a \textit{transaction}. All actions but the last represent work \textit{prior} to commitment, reading from data, creating temporary values or triggering an abort. If the last action is reached without aborting, the rule has \textit{fired} and aborts are no longer possible; the final action is then guaranteed to be executed, complete with any of the rule's observable effects. 

\subsubsection{Action Granularity}
Imperative form represents a protocol's defined interaction as actions to be computed in the specified sequence. At this stage, our representation is still symbolic; actions do not necessarily correspond 1-to-1 with concrete operations in the target imperative language, and their representation of actions is unspecified as long as they preserve the properties of imperative form. To avoid under-specification, we represent actions at the coarsest granularity possible to avoid \textit{over-specifying} the ordering of concurrent operations.

The simplest imperative form rules can be represented with a single action; implicitly, the rule has a trivial guard, and consists entirely of some guaranteed \textit{assignment}. For example, a rule with a trivial data constraint may be represented as a single, trivial action; the rule always fires, to no effect.

Connectors become more complex as they rely on the creation of temporary variables. For example, consider a protocol in RBF with data constraint $X=f(X)$ and synchronization constraint $\{X\}$ with only input (putter) port~$P$. This rule can be understood as ``$X$ fires \textit{if} the results of function~$f$ on its put-value is equivalent to the value itself''. Here, the result of $f$ clearly cannot be inspected until \textit{after} it is computed. We are able to represent this rule with an action sequence of length three: (1) Create temporary value $f_X$ by executing $f$ given argument $X$. (2) Trigger a rollback if $f_X\neq{}X$. (3) The rule has fired; do nothing other discarding values $X$ and $f_X$.


\subsubsection{Valid Value Access}
Imperative languages often prohibit using variables \textit{prior} to definition. Usually, the only case in which a variables access becomes invalid if it goes out of scope. Rust and affine languages generalize this notion, working to keep track of the moment when a value is \textit{consumed}, after which its access is invalid. 

For our purposes, it is useful to allow values to \textit{become invalid}. In the context of imperative form, this allows any actions to \textit{empty} memory cells, as long as they still follow the rule (ie.\ they are able to \textit{undo} the emptying). The motivation behind this extension is Reo's ability to express the access of the value of a memory cell both before and after it is overwritten. We are able to represent these kinds of interactions with sequential actions if we are able to \textit{empty} a filled memory cell's contents elsewhere. 

Ultimately, we elaborate our notion of value validity to correspond more closely with the actions an affine-typed compiler would perform: in reading over actions from top to bottom, \textit{inaccessible} identifiers become \textit{accessible} when their values are created, and accessible identifiers become inaccessible if their values are explicitly emptied. A variable access is valid only if it was accessible at the end of the previous action. This formulation makes clear the need of some means of deciding which values are \textit{initially accessible}. Our solution is made apparent in the definition of IF to follow.

%The imperative form representation includes specifies that a temporary variable $f_X$ be computed and stored. By assumption~1, the decision of firing these rules must follow the creation of these variables, clearly putting these actions in $S_a$. This matches our intuition: the guard may be unsatisfied, requiring that the creation of the temporary variables be reversed, ie.\ the temporary values are discarded silently.
%
%A final elaboration on the sequence of actions is necessary to account for every memory cell whose value~$x$ is synchronously read and overwritten. Reo's internal representation inherits syntax from \textit{temporal logic} to disambiguate reads before and after overwrites for such cases: $m$ and $m'$ represent the current and `next' values respectively, with overwriting occurring in-between. The common solution to such problems in an imperative context is to first \textit{move} the value to a temporary variable. To this end, we define an action for \textit{swapping} the contents of two memory cells (which may be empty or full). Such an action is intuitive for most imperative languages, and can be trivially \textit{reversed} such that it may be included in $S_a$ if needs be.

%\subsubsection{Data Movement}
%The Reo compiler's internal representation represents all a rule's \textit{observable effects} as a set of assignments. Equivalently, assignments are a partial mapping from \textit{destination} (any persistent location able to accept values ie.\ \textit{getter} ports or empty memory cells) to their assigned value. Imperative form has in common that it represents all observable effects as these assignments in this concurrent fashion. Concretely, $S_b$ is always defined as a single, coarse-grained action specifying the destination of values in the result of the rule firing. However, it represents this mapping turned on its head; expressed as a mapping from \textit{source} (a set of values) to sets of destinations. This representation is chosen for its orientation making more easy to track the movement of each datum. The intuition is that this representation makes it easier to treat values as \textit{resources} that potentially require delicate management. Depending on the nature of the target language, it may be necessary to check the \textit{affinity} (see Section~\ref{sec:affine_type_systems}) of these values, as it becomes of paramount importance to control the number of destinations per value. As the imperative form is language-agnostic, it makes no attempt to rectify this problem itself, relying on the $T_{run}$ task to resolve the problems in the manner befitting the target language. As an example, the Rust language is concerned with the \textit{affinity} of some its data types, and must explicitly create new affine resources with an explicit \code{clone} operation. This is discussed further in Section~\ref{sec:translation_phase_2} to follow.
%
%This movement representation also makes clear the cases for which a value goes \textit{unused}. \textit{Relevant} or \textit{linear} type systems may wish to reject such protocols, or insert explicit operations to handle the equivalent effect, as their definition forbids the destruction of data elements\footnote{For example, a linear imperative language may wish to simulate data destruction by injecting the special \code{Destroyed} destination for otherwise empty destination sets.}. Languages with explicit \textit{memory management} may need to address the cases of data destruction with the injection of special handlers. For example, the C language may necessitate the injection of a \code{free} call to avoid leaking memory.

\subsection{Definition}
\label{sec:imperative_form_definition}
Here we define \textit{imperative form} (`IF') concretely, and explain how its definition corresponds with the intuition behind it. Firstly, an IF contains a structure which corresponds to a \textit{symbol table}; this does the work of assigning symbolic \textit{data types} to ports and memory cells. Ports must also be annotated with an explicit \textit{orientation} (ie.\ input or output). Other symbols are also represented here, for example, the names and the argument types for any named functions.

More interesting are the \textit{imperative rules} listed for an IF. Each rule is given by $(P,I,M)$ where:
\begin{enumerate}
	\item \textbf{Premise $P$}\\
	A tuple of three \textit{identifier} sets $(P_R, P_F, P_E)$. $P_R$~is the \textit{synchronization constraint}, ie.\ the set of ports identifiers whose values must be `ready'. $P_F$ and $P_E$ are the sets of \textit{memory variables} which must be known to be full and empty respectively, such that it is known whether they can be read or written from. The rule can certainly not consider firing unless all ports are ready and all memory cells are in the specified states.
	
	\item \textbf{Instructions $I$}\\
	A list of reversible \textit{instructions} which are performed in sequence. These instructions have no immediately observable effects, such that they can be reverted in the event of a \textit{rollback}. Concretely, each instruction is one of:
	\begin{itemize}
		\item $check(p)$\\
		Trigger a rollback if predicate $p$ over data is satisfied.
		\item $fill_P(m, p)$\\Fill an empty memory variable $m$ with the result of a predicate $p$ over available data. The value's data type is implicitly \textit{boolean}.
		\item $fill_F(m, f, a)$\\
		Fill an empty memory variable $m$ with the result of invoking function $f$ with parameters $a$, a list of references to data variables with length matching the arity of $f$. It is incorrect for $f$ to \textit{mutate} its arguments, as this would result in observable effects which cannot be rolled back.
		\item $swap(m_0,m_1)$\\
		Swap\footnote{In principle, any reversible data-agnostic manipulation is possible, but swapping values is sufficiently expressive and intuitive for our purposes.} the values in two memory variables~$m_0$ and~$m_1$.
	\end{itemize}
	If a rollback is triggered by $check$, any swapped memory cells are swapped back, and any memory cells whose values were created by $fill_P$ or $fill_F$ are destroyed.
	
	\item \textbf{Movements $M$}\\
	A mapping from identifiers of \textit{values} to the identifiers of getter ports and empty memory cells. This represents the final action of an imperative rule executed if and only if the rule \textit{fires}.
\end{enumerate}

Our definition represents an elaboration of the underlying concept. $P$ and $I$ contain only \textit{transient} actions, which have no immediately-observable effects, and are able to handle the rule aborting such that their actions are \textit{undone}. $P$~is distinguished as it serves a dual purpose: (1) it establishes which values are initially \textit{accessible}, and (2) it tersely expresses an immediate conditional abort in the event the ports and memory cells are not in the states defined. $M$~is the final action, performed if and only if the rule commits. It defines what happens to all of the \textit{accessible} values still `in scope' at the end of the rule's execution. $I$~contains everything else, representing an arbitrary sequence of transient computation. These actions are subject to handling the rule being aborted, and thus our definition includes only reversible actions. Data is prohibited from being irrecoverably lost during $I$-actions, as otherwise the actions could not be \textit{undone} in the event of the rule aborting. This has three notable consequences:
\begin{enumerate}
	\item $fill_F$ can only rely on \textit{pure} functions, ie.\ their execution in and of itself must not be observable to the outside world.
	\item $fill_F$ and $fill_P$ can only write to \textit{inaccessible} values, ie.\ they cannot overwrite accessible values.
	\item Accesses of any values must not mutate or consume the original, eg.\ using a value as an argument in $fill_F$.
\end{enumerate}

%This definition represents an elaboration of the underlying concept. Rather than a single list over actions constrained by the properties they must satisfy, we make explicit three \textit{classes} of actions such the properties are structurally guaranteed for any well-formed IF. Despite them comprising transient actions evaluated in the order of their appearance, $P$ is distinguished from $I$ as it allows a terse, ergonomic means of both (1) making explicit which values are accessible
%
%
%$P$ and $I$ both represents \textit{transient} actions prior to commitment; $P$ is distinguished, allowing us to simultaneously define which \textit{persistent values} (values from the global scope, ie.\ not temporary values) are accessible and known to be 
%
%Conceptually, this definition represents a minor elaboration on the underlying concept. Deviations are motivated primarily by convenience. P
%\begin{enumerate}
%	\item The final, committing action is provided its own name, $M$, to disti
%\end{enumerate}
%
%Conceptually, this definition adheres closely to the intuitive description of imperative form. To reflect the special treatment of the list of actions, we here explicitly distinguish it into $M$. $P$ and $I$ are distinguished to make clear the set of values that the imperative rule must access from the global scope; implicitly, all values in $I$ must be temporary and thus are not permitted to overwrite 

As an example to demonstrate this representation, the RBA rule in the previous section with data constraint $X=f(X)$
and synchronization constraint $\{X\}$ can be represented in the imperative-form rule with:

\vspace{1em}
\noindent{}
\begin{tabular}{r|l}
	\centering
		&  value \\ \hline
	premise	&  $(\{X\}, \enskip\{\}, \enskip\{f_X\})$ \\
	instructions	& $[fill_F(f_X, f, [X]), \enskip{} check(X = f_X)]$ \\
	movements	& $\{X \rightarrow{}\emptyset{}, \enskip f_X \rightarrow{}\emptyset{}\}$ 
\end{tabular}
\vspace{1em}

\section{Translation Pipeline}
In this section be describe how Reo's internal representation (`IR') of protocol specifications is translated to a Rust protocol object. As per the design in Section~\ref{sec:two_phase}, this process involves generation steps partitioned over two distinct phases with imperative form (`IF') in-between. Here, we describe this process beginning to end. Throughout this section, we refer to the generation process in terms of its three distinct subtasks $\{T_{seq}, T_{type}, T_{run}\}$, defined in Section~\ref{sec:sub_tasks}.

%\subsection{Temporary Simplifications}
%IF includes explicit annotations for the data types of ports, memory cells and temporary variables, representing the completion of $T_{type}$. Our definition thus-far reflects our conceptual design, where the translation to IF is entirely target-language agnostic. In our current implementation, this is not strictly true; our translation to IF is not entirely free of Rust for one purpose only: currently, Rust is the only target language making use of IF. For the purpose of minimizing the (uninteresting) implementation work for the project, the Reo compiler emits rust source directly, which is a thin, Rust-friendly wrapper over a data structure of a protocol specification in IF as one would expect. This simplification has three advantages:
%
%\begin{enumerate}
%	\item We trivialize the burden on the second translation phase by representing IF in a form the Rust compiler understands inherently: a type in the Rust language. 
%	\item We can rely on Rust's \textit{generic types} to generate Rust dependency which is able to delay the resolution of its symbolic data-types (part of $T_{run}$) to the last possible moment, relying on a representation that is both idiomatic for the language, and trivial to implement: we represent them as Rust \textit{generic types}.
%	\item We do not deviate from the current idiom in the Reo compiler, whereby a target language generates a dependency for the source language directly.
%\end{enumerate}
%
%Effectively, our current implementation of Reo's Rust backend emits a Rust-specific dependency which relies almost entirely on its IF intermediate form in the manner described in Section~\ref{
%}

\subsection{Reo Compiler Backend}
\label{sec:translation_phase_1}
We extend the Reo compiler with a backend for translating IR to Rust source on which the end user may depend. In this phase, the work is primarily concerned with restructuring IR to IF, involving two of the three sub-tasks for generating to a particular imperative language target, namely~$T_{seq}$ and~$T_{type}$. 

\subsubsection{Action Sequencing}
$T_{seq}$ necessitates transforming a each of the protocol's rules into a sequence of symbolic actions. The most significant work occurs as a result of how differently \textit{values} are represented. IR is declarative, representing the result of a rule's firing as an \textit{assignment}, mapping \textit{destinations} (getter ports and empty memory cells) to \textit{terms}. IR already represents a significant transformation from RBF in isolating these values on a per-destination basis. 

To begin, we describe the na\"ive approach to translate IR to IF, one rule at a time.

Our translation procedure initializes all three fields $\{P, I, M\}$ of an imperative rule as initially-empty, and populates them incrementally by recursively traversing the IR rule's assignments. Each such assignment is ultimately represented in~$M$, where \textit{terms} are rather represented by identifiers. For some terms the mapping to identifier is trivial. For example, the value put by a port can use the identifier of the port itself. For others, it may be necessary to introduce \textit{fresh} identifiers, representing \textit{temporary variables} to be created. In either case, the \textit{term} is traversed recursively to (1) collect these identifiers, and to (2) populate the premise~$P$ such that the rule is fired given access to all of the relevant memory cells and ports.

$I$ is populated last. Firstly, the exceptional cases for which memory variable~$q$ will be both read and written to are treated. If necessary, a fresh temporary variable is introduced by appending an instruction $swap(q, q_{old})$ where $q_{old}$ is some fresh variable; $q$'s previous and next values may be read and written unambiguously, distinguished by identifiers $q_{old}$ and $q$ respectively. Second, $I$ is appended with an instruction to create every other temporary variable in a manner befitting the \textit{term} that represented them in the IR, ie.\ the result of invoking a function with port values as arguments. Finally, $I$~ends with a $check$ to evaluate the rule's guard, initiating a rollback if it is not satisfied.

%The procedure as it was described thus-far may be 

%has a defined creation instruction
%created in a manner befitting their associated \textit{term};
%
%
%%Ultimately each assignment is represented in~$M$, mapping the identifier of a value to the identifier of the \textit{destination}. For the majority of cases, these terms are trivial, representing the identifier of one of the protocol's memory cells or ports. In all cases, the backend's task is to 
%%
%%
%%
%%
%%
%%For each assigned term, the backend computes the \textit{identifier} which will contain the associated value once the rule \textit{commits}. Trivial terms are \textit{primitives}, corresponding directly to memory cells or ports. For these cases, the identifier of such is used directly, resulting only in an update to~$P$ to
%%
%%The term at each level is understood as some \textit{value}. The simplest terms are primitive (ie.\ the value \textit{put} by a port or the contents of a filled memory cell), and correspond with values that can be included in~$P$ as part of the premise. All other terms represent temporary values, which are associated with fresh \textit{identifiers}. 
%%Primitives are the vast majority in practice, and are simply associated with the identifiers of their port or memory cell. All others are assigned fresh \textit{temporary variable} identifiers, their order in the collection reflects the order in which they were encountered. 
%
%The most na\"ive means of translating IR to IF is achieved by inserting instructions into~$I$, creating all temporary variables, careful to order these instructions such that their corresponding terms always involve created values, ie.\ leaf terms occur above their roots in~$I$. The final instruction is a single monolithic \textit{check} instruction, which triggers rollback at the last possible moment, safe in the knowledge that any values accessed during evaluation already have been created. Finally,~$M$ is populated with mappings of movements; every identified value is included, with a \textit{destination} for every port or memory cell in whose assignment-term the value occurred as mapped the root term. Any putter-ports in the synchronization constraint that never occur in terms are provided with trivial mappings in $M$ to ensure their values are acquired and discarded correctly.
%A final elaboration is necessary for every memory cell that is both read and written to. This occurs whenever it occurs both as a \textit{key} in the assignment map (it is assigned to, ie. written) and as a subterm of some value (ie.\ its value is accessed directly or indirectly in an instruction, eg.\ it occurs as an argument to a function call, computing a temporary value). For each such memory cell~$q$, $I$ is prefixed with a $swap(q, old_q)$ instruction, where $old_q$ is fresh, initially empty temporary variable. For all instructions to follow, and in~$M$, $q$~acts as a getter of the memory cells written value, while $old_q$ acts as a putter corresponding with a value moved to some nonempty set of destinations.

As it was described thus-far, our procedure is able to correctly render any IR rule in IF with the necessary properties. For the sake of minimization or performance at runtime, at least three optimization opportunities may elaborate on this procedure, producing semantically-equivalent results.

\begin{enumerate}
	\item Terms that occur repeatedly within assignments throughout the same IR rule may have their \textbf{values deduplicated} by assigning them all the same \textit{identifier}. Care must be taken to ensure that the instruction to create its value is inserted only once, sufficiently early that its creation precedes its \textit{earliest} access. Note that each original occurrence still corresponds with a \textit{destination} in the resultant~$M$ mapping. To clarify, consider the example with getter ports~$A$ and~$B$ both assigned terms corresponding to $f(C)$ where $f$ is some function and $C$ is a putter port. Here, one temporary variable $f_C$ to store the result of $f(C)$ is sufficient; it is simply moved to two distinct destinations, reflected in the mapping $f_C\rightarrow\{A,B\}$ in~$M$.
	
	\item The large, monolithic \textit{check} instruction that acts as a guard to the rules firing can be fragmented into \textbf{numerous guard instructions}. The utility of this is the ability to re-arrange their ordering. For best results, it is beneficial to move checks as early as possible, such that less work is performed prior and subsequently to a rollback whenever the check \textit{fails} at runtime. To be correct, care must be taken not to move guards so early such they precede the creation of any temporary variables their evaluation accesses. For example, consider an IR 
	whose guard is $A\wedge{}B$, where~$A$ and~$B$ are sub-formulas that reason about sub-terms whose evaluation necessitates the creation of temporary values~$t_A$ and~$t_B$. By fragmenting $check(A\wedge{}B)$ into $check(A)$ and $check(B)$, we are able to move the former such that it follows the creation of $A$, but not of $B$. Effectively, the rule is able to \textit{short circuit} its evaluation at runtime, circumstantially avoiding the creation and destruction of the temporary value identified by~$t_B$.
	
	\item \textbf{Static analysis} of values may conclude that a \textit{check} instruction is a tautology, making it safe to omit. Similarly, the presence of even one contradictory \textit{check} makes it possible to discard the rule entirely. This optimization is particularly useful in combination with optimization (2).
\end{enumerate}

\subsubsection{Type Classification and Constraining}
Our backend performs task $T_{type}$ to generate the IF such that the identifier of every port, memory cell, and temporary variable is assigned a symbolic type annotation, such that:
\begin{enumerate}
	\item the types of identifiers match if they exchange values or are checked for equivalence.
	\item no function has contradictory requirements on the types of its arguments.
	\item identifiers used in a boolean-only context (ie.\ in the root of a \code{Formula}) are boolean.
	\item the type defines all the operations in which it may be involved at runtime. This includes operations for \textit{replication}, checking \textit{equality} of values and so on.
\end{enumerate}

Our backend performs this work in tandem with the work of $T_{seq}$ described in the previous section. Initially, every identifier is assigned s fresh symbolic type with no constraints, representing a data-type unrelated to any other, and having no need of any defined operations. In traversing the IR rules, \textit{constraints} are collected, associating them to the relevant identifiers. Incrementally, \textit{type constraints} are collected, accumulating requirements for the properties of types. For example, the type associated with a value is checked for equality, irrevocably acquiring the \code{eq} constraint, marking the need for it to define an operation to check value-equality. In some cases, a relationship between identifiers causes their types to be \textit{unified}. For example, a data-movement from putter~$P$ to getter~$G$ unifies their types, resulting a new type with the union of their constraints.

Ultimately, an IF constructed with a global \textit{symbol table}, defining the mapping from identifiers to types such that our requirements are satisfied.

%IF expects an explicit type-annotation for the \textit{identifiers} of every port, memory cell and temporary value. As IR cannot be relied upon to create explicit annotations, our backend is reponsible for $T_{type}$, the resolution of all symbolic types for identifiers. As IF is still a language-agnostic specification, these types serve little purpose other than to act as \textit{equivalence classes} and discover which properties these symbolic types much satisfy such that the target language can ensure all the 
%
%Previously referred to as $T_{type}$, the backend must extract data-type information about its ports and memory cells from the compiler's internal representation such that the generated Rust is valid. Our approach is to begin by assuming that every port, memory cell and temporary variable has its own unrelated type. The types are then \textit{constrained} as a result of discovering the ways in which they are related from the protocol definition. For example, the presence of a term \code{Eq(A,B)} `collapses' the types of~$A$ and~$B$ into one equivalence class. Still, at this level, types are purely symbolic. 
%
%These symbolic types exist only in the first phase of the code generator. They are not only resolved prior to phase two, but they are not present in that phase at all. Our imperative form does not deal with the complexity of symbolic (ie.\ generic or parametric) types. Instead, the Reo compiler relies on the Rust idiom of relying on \textit{dispatch} to resolve concrete types at the last possible moment: at the call site. To achieve this result, Reo generates the \code{ProtoDef} object wrapped in a \textit{wrapper function}, whose role is primarily to expose generic type parameters for instantiation to the caller, and construct a \code{ProtoDef} instance within the body, to be invoked with concrete types. Listing~\ref{listing:type_resolve} gives an example of how this generic function appears to the end user for some trivial protocol. Observe how \code{protocol\_1} relies on generics to let the caller, \code{main\_1} which type to select for~\code{T}. The \code{TypeInfo} structure represents a port's data type as data. The details of this type are provided in Section~\ref{sec:type_reflection}.
%
%
%\begin{listing}[ht]
%	\centering
%	\inputminted[]{rust}{type_resolve.rs}
%	\caption[Concrete vs.\ generic protocol-building functions.]{Comparison between concrete and generic function definitions for building \code{ProtoDef} structures. \code{new\_proto\_a} uses the concrete \code{String} type, and will be compiled to a single function as expected. Reo uses the approach of \code{new\_proto\_b}, which determines the choice of the generic type on demand at the \textit{call site}.}
%	\label{listing:type_resolve}
%\end{listing}
%
%Unlike Java, Rust makes very few promises about the properties of some generic type. It cannot be certain that instances of some generic type~\code{T} will have a defined operation for checking equality, or for safe \textit{value replication}. To facilitate \textit{useful} polymorphism, Rust relies on \textit{type constraints} to act as a contract between caller and callee: the caller will ensure that only types which satisfy the bound may be selected for the type parameter, and the callee is able to interact with its generic arguments in accordance with \textit{behavior} the bounds guarantee the type will define. This approach may be familiar to programmers of Java or C, where this concept manifests as interfaces and declarations (usually in header files) respectively. 
%
%The Reo backend cannot guarantee the generated Rust code is sound unless it is careful to add the necessary type bounds to its generic arguments. The internal representation is inspected for cases which will necessitate the use of specialized operations (such as replication) and will annotate its generic types with \textit{trait bounds}. Ultimately, the resulting coalesced, bounded type parameters are reflected in the generated code as part of the function declaration. The second phase of the code generator can rest assured that for every specialized operation inserted, Reo will have already anticipated the need to guarantee the bound is satisfied. Listing~\ref{listing:type_resolve2} gives an example of a generated trait bound for the case where two memory cells are related by being the same, and requiring the definition of an equality operation.
%
%\begin{listing}[ht]
%	\centering
%	\inputminted[]{rust}{type_resolve2.rs}
%	\caption[Reo-generated trait bounds for generic types.]{Reo-generated \code{ProtoDef} building functions with a generic type~\code{T}. Reo inserts trait bounds to ensure the type chosen for~\code{T} has all the needed behavior. In this case, \code{T:~Eq}, ensuring that instances can be checked for equality, as necessitated by the instruction on line~9. Line~16 generates a compile-time error, as~\code{Foo} does not meet the requirements.}
%	\label{listing:type_resolve2}
%\end{listing}

\subsubsection{Delegated to the Rust Compiler}
Section~\ref{sec:temporary_simplifications} explains that our current implementation of the Rust backend for the Reo compiler makes the temporary simplification of emitting Rust source code directly. This approach adheres with the Reo compiler's idiom of code generation per target language, but also it simplifies our work overall as we are able to effectively \textit{delegate} some of the translation work to the Rust compiler itself. Both of these simplifications are inspired by Reo's Java code generator, whose direct-to-Java code generator delegates these tasks to the Java compiler in the same manner:
\begin{enumerate}
	\item \textbf{Parsing}\\
	Per our design, IF should be emitted in a format agnostic to the imperative language target; good contenders are common data serialization formats such as JSON or YAML. Ultimately, IF is translated to the syntax understood by the target language such that it can be integrated into the user's programs. While Rust is the only language target, we are able to unify these steps by emitting Rust syntax as a usable dependency directly.
	
	\item \textbf{Type Resolution}\\
	Our symbolic data types are exposed as \textit{generic types} in the emitted source; effectively, the end-user's Rust compiler makes concrete these symbolic types at the call site, as is idiomatic in the Rust language.
\end{enumerate}

The second task is most interesting, as care must be taken to represent our generic types in a manner that the Rust compiler will accept. Previously, we described how requirements on our symbolic types are discovered throughout $T_{type}$. Here, these constraints are communicated to the Rust compiler in the generated syntax. This delegates the task of enforcing these constraints to the end-user's Rust compiler. Listing~\ref{listing:generic_resolve} gives an example of a signature for a Reo-generated Rust function with constrained generic types~\code{X} and~\code{Y}. Observe that the majority of the functions contents are the definition of a~\code{ProtoDef} type, which is the Rust-embedding of our IF. 

\begin{listing}[ht]
	\centering
	\inputminted[]{rust}{generic_resolve.rs}
	\caption[]{}
	\label{listing:generic_resolve}
\end{listing}

%Consequently, there are two minor discrepancies between the concept and its implementation in practice:
%\begin{enumerate}
%	\item The initial
%\end{enumerate}
%
% 
%IF is a specification focused on the \textit{behavior} of a connector at runtime. Notably, it omits 

%\subsubsection{Initial Protocol State}
%The initial state of a protocol object is unusual in that it is included in the textual Reo protocol definition, but omitted from the \code{ProtoDef}. This design choice reflects how a protocol's initial state is unassuming at first glance, but is significantly different from the rest of the protocols definition: it is the only part of specification that cannot generally be replicated. Rules and name definitions describe \textit{behavior}, and do not involve any elements of the port's data domain directly. By contrast, a protocol's initial state does not \textit{describe} data, it \textit{is} data. By extracting this facet of the specification, \code{ProtoDef} becomes pure in its role as a \textit{blueprint} for a protocol's behavior. 
%
%Nevertheless, textual Reo is able to specify the initial states of memory cells. To support this functionality, the Reo compiler itself generates code which \textit{builds and injects} the initial values for any initially-filled cells in a controlled environment. To mirror the textual descriptions that define the initial values, these memory types acquire a final trait bound such that their initial values can be constructed per protocol object instantiation \textit{within} the confines of the function that the Reo compiler generates. Listing~\ref{listing:mem_init} exemplifies a trivial protocol where~$A$ is a memory cell defined as being initialized by some value represented by the string `\texttt{hello}'. The generated code inserts a trait dependency, ensuring that the type parameter~\code{T} defines this operation. Observe how now the return result is no longer~\code{ProtoDef}, but rather~\code{ProtoHandle}. While the former represents a re-usable \textit{blueprint} for instantiating a~\code{ProtoHandle}, the latter represents an instantiated protocol, initialized and ready to run; the only user-facing means of generating a new object of the same protocol is to again invoke~\code{protocol}.
%
%\begin{listing}[ht]
%	\centering
%	\inputminted[]{rust}{mem_init.rs}
%	\caption[Reo-generated builder function, returning a protocol instance.]{\code{new\_protocol} instantiates a runnable protocol object by internally building a \code{ProtoDef} description, and then immediately instantiating it with a \code{MemInitial}, whose contents are constructed from parsed strings. In this manner, Reo can control the initialization of protocol instances for any suitable type~\code{T}. Note that \code{\&str} is the immutable reference type of a sized string slice, while \code{String} is an owned, mutable character buffer. Both are common types from the Rust standard library.}
%	\label{listing:mem_init}
%\end{listing}


\subsection{Rust Library}
\label{sec:translation_phase_2}
Our work follows the precedent set by the Java code generator in relying on a library in the target language to define the lion's share of the behavior for our runtime protocol objects. For Rust, these definitions are bundled into the \textbf{Reo-rs} library, which is added as a dependency to the code generated by the Reo compiler's backend. Chapter~\ref{sec:protocol_runtime} explores the architecture and behavior of our executable protocol objects in detail. For now, it suffices to say that our approach is to represent executable protocols as extensively preprocessed \textit{data structures} which then drive the behavior of a lightweight \textit{interpreter} at runtime. This data-representation is often called \textit{commandification}~\cite{nystrom2014game}. 


\subsubsection{Protocol Initialization}
Listing~\ref{listing:generic_resolve} gives an example of Reo-generated Rust source. Previously, we explain how this representation delegates some of the work of $T_{run}$ to the Rust compiler itself. The remainder of $T_{run}$ is defined in the \code{build} procedure (visible is the listing on line-23) such that this work is performed at runtime whenever a protocol is \textit{instantiated}. As can be seen in the \textit{return value} of the function in the listing, the end result is the construction of \code{Proto}, the \textit{executable} protocol object, whose properties are explored in Chapter~\ref{sec:protocol_runtime}.

At this level, the specification used to \code{build} the \code{Proto} is split into two distinct types: \code{ProtoDef} and \code{MemInitial}. The former describes its \textit{behavior}, corresponding most closely with the conceptual design of IF. The latter isolates a simple structure which contains pre-allocated values for the finished \textit{protocol object}; in effect, it provides the initial values of memory cells. These structures are distinguished for only one reason: \code{ProtoDef} contains no \textit{values}, such that it can be accessed by \code{build} in a read-only fashion. Although it is not taken advantage of by the Reo-generated program, one is able to define a protocol's behavior \textit{once} to be used for the construction of any number of \code{Proto} instances.

\code{Proto} represents the ultimate departure from its original, declarative protocol specification whose purpose is to facilitate execution. Its creation from \code{(ProtoDef, MemInitial)} involves the last remaining sub-tasks of $T_{run}$:
\begin{enumerate}
	\item \code{Proto} is constructed along with data structures necessary for basic operations at runtime. This includes semaphores, channels for \textit{control messages} and so on. This minutia is detailed in Section~\ref{sec:protocol_object_architecture}.
	
	\item In construction, the behavioral specification (ie.\ imperative form rules) are \textit{preprocessing}  to a form more conducive to efficient execution. For example, symbolic names are replaced with indices, pointers and keys into concrete data structures. The \textit{soundness checks} in the section to follow can be considered preprocessing also, as they ensure a \code{Proto} is constructed such that it is internally-consistent, ensuring various runtime properties are invariant and need not be checked at runtime.
\end{enumerate}


\subsubsection{Soundness Check}
Our backend is novel in that the work of constructing the executable protocol object requires crossing an API-boundary. Rather than trusting the well-formedness of the Reo-generated \code{ProtoDef}, Reo-rs will check that its input is internally-consistent. By adding these checks, the dependency between the Reo compiler and Reo-rs is \textit{unidirectional}; users are free to safely construct protocol objects using Reo-rs in their applications directly. The most obvious advantage to this approach is an additional layer of safety, allowing for the compiler and Reo-rs to be maintained separately, eg.\ if the compiler acquires a bug from a new update, the error cannot propagate into Reo-rs unnoticed. Another advantage is the avenues for future work this opens up. Our approach treats protocol structures as data, facilitating their mutation at runtime, resulting in \textit{dynamic protocol reconfiguration}, although exploring this further is beyond the scope of this work.

In native Rust, the usual variable scoping rules apply to ensure that a symbolic identifier is resolved to a meaningful memory position. These systems are so familiar to us that we usually take the complexity of their work for granted. Rules that are second nature to us require explicit enforcement to replicate the work of the checker. As \code{ProtoDef} \textit{commandifies} the behavior to be later executed, the Rust compiler does not interpret these actions in the normal way, and we must mimic its behavior manually. We take this idea a step further by relying on the \code{ProtoDef}'s \textit{premise} to facilitate a mechanism that mimics the Rust \textit{borrow checker} system; rules trace which variables are \textit{valid} (ie.\ initialized) throughout the rule's execution top to bottom, tracking changes as a result of actions filling or swapping their values. In this manner, we are able to catch invalid memory accesses during \code{build}, rather than encountering them at runtime. An additional perk of mimicking this system is our ability to detect the occurrence of values which \textit{must} be consumed during the rule's firing, but whose consumption is not included in the specification. For example, an \textit{imperative rule} may include some putter port $P$ in its ready-set (and thus, its synchronization constraint), but associate no \textit{movement} with $P$'s value. A na\"ive implementation which overlooks such occurrences may introduce \textit{memory leaks} for such cases if it takes the specification at face value. Instead, our custom borrow checker will reach the end of of the specified actions and conclude that as $P$ was not explicitly emptied, it will insert a trivial movement $P\mapsto{} \{\}$ to ensure the value is consumed. This is analogous to how Rust's borrow checker inserts \code{drop} calls to destroy local variables which go out of scope unconsumed. By performing this extensive checking, Reo-rs affords an expressive \code{build} function, capable of giving detailed error information in response to an invalid input. The signature of this function is given in Listing~\ref{listing:build}.


\begin{listing}[ht]
	\centering
	\inputminted[]{rust}{build.rs}
	\caption[TODO.]{Signature of the~\code{build} function. Its inputs are (1) an immutable reference to a \code{ProtoDef}, which is used to determine the protocol's behavior, and (2) a \code{MemInitial}, which stores initialized memory cells to be incorporated into the protocol's state. The return result is an enumeration type, returning \code{ProtoHandle} upon success, and a tuple on failure, whose elements are, respectively (1) the index of the imperative rule where the error occurred if applicable, and (2) another sum type, communicating the nature of the error with additional information. }
	\label{listing:build}
\end{listing}

By restricting our API such that all executable protocol objects are \textit{only} created by \code{build}, our runtime interpreter is able to rely on the properties we guarantee and avoid checking them itself. In this way, checking for soundness is also an optimization.

