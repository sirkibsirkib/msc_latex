\FloatBarrier
\begin{listing}[ht]
	\centering
	\inputminted{rust}{get_data.rs}
	\caption[Getter procedure for acquiring resource data.]{A getter's procedure for retrieving a value from a putter or memory cell. Getters must coordinate such that one is elected the \textit{mover} with all others cloning. The mover must go last, and once everyone is done, the resource must be cleaned up.}
	\label{listing:get_data}
\end{listing}

\begin{listing}[ht]
	\centering
	\inputminted{rust}{alternator_manual.rs}
	\caption[Hand-crafted alternator implementation.]{Hand-crafted alternator implementation ini Rust based on channels from the \code{crossbeam} crate and a standard-library \code{Barrier} for explicit synchronization. This simple design is chosen for its simplicity and its close correspondence to the Reo channels that constitute its specification.}
	\label{listing:alternator_manual}
\end{listing}


\begin{listing}[h!]
	\centering
	\inputminted{text}{mpsc_pop.txt}
	\caption[x86-64 assembly of a standard Rust channel, showing in-lining.]{Snippet out of the x86-64 assembly generated by receiving a large datum through \code{recv} from a simple channel from the Rust standard library. It unrolls the movement of the entire object into a large sequence of smaller operations rather than invoking a system call. This is the case for the receipt of a \code{Copy}-type represented by 512 bytes.}
	\label{listing:mpsc_pop}
\end{listing}




\begin{listing}[h!]
	\centering
	\inputminted{rust}{refl_test_in.rs}
	\caption[TypeInfo creation example input Rust.]{Example of how the \code{TypeInfo::of} function (1) ensures the compiled binary includes a vtable for the requested type~\code{T} with \code{PortDatum} as its interface, and (2) returns the pointer to the vtable.}
	\label{listing:refl_test_in}
\end{listing}


\begin{listing}[h!]
	\centering
	\inputminted{text}{refl_test_out.rs}
	\caption[TypeInfo creation example output assembly.]{x86-64 assembly of Listing~\ref{listing:refl_test_in}. \code{.L\_\_unnamed\_1} shows the binary representation of the vtable of the \code{u32} (32-bit unsigned integer) type and \code{PortDatum} trait. Rust's vtables have a predictable structure with three fields followed by trait-defined function pointers. Lines~10--12 store the concrete type's (1) \code{drop} function pointer, (2) size in bytes, (3) memory-alignment in bytes. }
	\label{listing:refl_test_out}
\end{listing}
